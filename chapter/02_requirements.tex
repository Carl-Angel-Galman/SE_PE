\section{Requirements Recap (Kurzfassung)}

\subsection{Funktionale Kernanforderungen}
\begin{itemize}[leftmargin=*, itemsep=2pt]
  \item \textbf{Hard Stop (Pilzbutton):}
  Durch Betätigung des \emph{Emergency Button (Pilzbutton)} muss das System unverzüglich einen \emph{Hard Stop} auslösen.

  \item \textbf{Übergang nach Lösen des Pilzbuttons:}
  Wird der Pilzbutton gelöst, muss das System aus dem \emph{Hard Stop} in einen \emph{Soft Stop} wechseln.

  \item \textbf{Soft Stop lösen:}
  Der \emph{Soft Stop} muss über den \emph{Soft-Stop Button} sowie über die externe \emph{Mission Control} gelöst werden können.

  \item \textbf{Soft Stop durch Mission Control:}
  \emph{Mission Control} muss einen \emph{Soft Stop} initiieren und anschließend auch wieder lösen können.

  \item \textbf{Softwareseitiger Stop basierend auf TTC:}
  Das externe \emph{TTC}-Modul liefert die \emph{Time-To-Collision} als Entscheidungsgrundlage.
  Bei entsprechender Bewertung muss das System einen softwareseitigen Stopp auslösen, der als \emph{Soft Stop} umgesetzt wird.

  \item \textbf{Logging:}
  Alle stop-relevanten Ereignisse (Auslösen, Zustandswechsel, Lösen) müssen zuverlässig geloggt werden.
\end{itemize}