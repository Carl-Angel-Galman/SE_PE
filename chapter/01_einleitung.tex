\section{Einleitung}

\subsection{Ziel und Zweck}
Dieses System Design Document beschreibt die Architektur und das detaillierte Design der Subkomponente
\emph{Emergency Stop} im Truck-System. Die in einer vorgelagerten Phase erstellte SRS wird als Grundlage vorausgesetzt; dieses Dokument konkretisiert diese
Anforderungen in einem umsetzungsnahen Software-Design.

\subsection{Einordnung im Gesamtsystem}
Die Subkomponente \emph{Emergency Stop} ist eine sicherheitsrelevante Funktion innerhalb des Truck-Gesamtsystems.
Sie verarbeitet Eingaben aus externen Systemen und von Hardware-Komponenten und leitet daraus Stopp- und Signalisierungsaktionen ab.

\subsection{Funktionsprinzip (Überblick)}
Das \emph{Emergency Stop}-Verhalten basiert auf zwei zentralen Auslöseprinzipien:
\begin{itemize}[leftmargin=*, itemsep=2pt]
  \item \textbf{Hardware-Auslösung:} Der Pilzbutton führt zu einem sofortigen Stopp (Hard Stop). Beim Lösen des Pilzbuttons wird
        in einen Soft Stop übergegangen, der anschließend gezielt gelöst werden kann.
  \item \textbf{Software-Auslösung:} Das TTC-Modul liefert die \emph{Time-To-Collision} als Entscheidungsgrundlage für einen
        softwareseitigen Stopp (Soft Stop). Im \textbf{Normalbetrieb} befindet sich das System in der \emph{Free Zone}.
        Sinkt der TTC-Wert in den \textbf{kritischen Bereich} (rote \emph{Safety Zone}), sendet das TTC-Modul eine \emph{Stop Request}
        an die Subkomponente \emph{Emergency Stop}. Diese führt den Stopp entsprechend der Zustandslogik aus.
\end{itemize}

\begin{figure}[H]
  \centering
  \includegraphics[width=\textwidth]{pictures/Abstand.png}
  \caption{Illustration des Zonenmodells als Grundlage TTC-basierter Entscheidungen (Normalbetrieb vs. Intervention/Safety)}
  \label{fig:zones_ttc}
\end{figure}

\subsection{Kontextsystem}
Die Subkomponente \emph{Emergency Stop} wird auf der Truck-Plattform betrieben und interagiert mit der Umgebung über klar
abgegrenzte Schnittstellen (externe Systeme) sowie über Ein-/Ausgabe-Hardware (Buttons/LEDs/Warnleuchten).
Die nachfolgende Abbildung dient als anschaulicher Kontext für die betrachtete Plattform.

\begin{figure}[H]
  \centering
  \includegraphics[width=\textwidth]{pictures/TruckBild.png}
  \caption{Truck-Plattform als Kontextsystem der Subkomponente \emph{Emergency Stop}}
  \label{fig:truck_platform}
\end{figure}

\subsection{Scope und Abgrenzung}
Der Scope dieses Dokuments umfasst ausschließlich die truckseitige Subkomponente \emph{Emergency Stop} inklusive ihrer
Schnittstellen, der internen Architekturstruktur sowie der detaillierten Ausgestaltung des Verhaltens.
Andere Komponenten des Truck-Systems werden nicht entworfen und nur soweit erwähnt, wie sie als externe Nachbarn oder
Schnittstellen für \emph{Emergency Stop} relevant sind.

\subsection{Systemgrenze und externe Schnittstellen}
Folgende Nachbarsysteme, Schnittstellen und Hardware-Elemente werden als \textbf{extern gegeben} angenommen und sind nicht Bestandteil des
internen Designs (sie werden in den Diagrammen als externe Komponenten bzw. externe Hardware dargestellt):

\begin{itemize}[leftmargin=*, itemsep=2pt]
  \item \textbf{Externe Software-/Systemschnittstellen:}
  \emph{Mission Control}, \emph{Environment Control Interface}, \emph{TTC}, \emph{Logging}, \emph{Drive Control Unit}
  \item \textbf{Externe Hardware:}
  \emph{Emergency Button (Pilzbutton)}, \emph{Soft-Stop Button}, \emph{Warnleuchten}, \emph{Soft-Stop LED}
\end{itemize}