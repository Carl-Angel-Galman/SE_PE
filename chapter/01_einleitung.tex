\section{Einleitung}

\subsection{Ziel des Dokuments}
Dieses System Design Document beschreibt die Softwarearchitektur sowie den detaillierten Entwurf (Detailed Design) der
\emph{Emergency-Stop}-Funktionalität im Truck-System. Grundlage ist die bereits erstellte SRS, die als gegeben angenommen wird.
Das Dokument dient als technische Spezifikation für die Implementierung und zur Sicherstellung der Konsistenz zwischen Anforderungen,
Architektur und UML-Modellen.

\subsection{Scope und Abgrenzung}
Der Scope dieses Dokuments umfasst ausschließlich die truckseitige Software für den \emph{Emergency Stop} inklusive
Zustandslogik, Schnittstellenkommunikation, Logging, Fehlerbehandlung sowie Ein-/Ausgabeanbindung (Buttons/LEDs).
Funktionalitäten außerhalb des Emergency-Stop-Kontextes werden nicht entworfen und nur dann erwähnt, wenn sie für die
Systemgrenze oder Schnittstellen notwendig sind.

\subsection{Systemgrenzen und externe Schnittstellen}
Innerhalb der Systemgrenze liegen die internen Subsysteme der Emergency-Stop-Software.
Folgende Komponenten werden als extern gegeben angenommen und sind nicht Teil des internen Designs:

\begin{itemize}[leftmargin=*, itemsep=2pt]
  \item \textbf{Externe Software-/Systemschnittstellen:}
  \emph{Mission Control}, \emph{TTC}, \emph{Logging}, \emph{Drive Control Unit}
  \item \textbf{Externe Hardware:}
  \emph{Emergency Button (Pilzbutton)}, \emph{Soft-Stop Button}, \emph{Warnleuchten}, \emph{Soft-Stop LED}
\end{itemize}

\subsection{Interne Subsysteme (Überblick)}
Die Emergency-Stop-Software wird in folgende interne Subsysteme gegliedert (Benennungen wie in den Diagrammen auf Englisch):
\begin{itemize}[leftmargin=*, itemsep=2pt]
  \item \textbf{EmergencyStopper} (u.\,a. TTC-basierte Entscheidung/Trigger)
  \item \textbf{StateMachineCommunication} (Kommunikation zu Mission Control und Drive Control Unit)
  \item \textbf{Logging} (Protokollierung stop-relevanter Ereignisse)
  \item \textbf{ErrorHandler} (Fehlererkennung und -behandlung)
  \item \textbf{LedHandler} (Ansteuerung Warnleuchten und Soft-Stop LED)
  \item \textbf{ButtonHandler} (Eingaben Pilzbutton und Soft-Stop Button)
\end{itemize}

\subsection{Sprache und Namenskonvention}
Das Dokument ist auf Deutsch verfasst. Klassen-, Subsystem- und Schnittstellennamen folgen den englischen Bezeichnungen
aus den UML-Diagrammen, um eine eindeutige Zuordnung zwischen Text und Modell sicherzustellen.
