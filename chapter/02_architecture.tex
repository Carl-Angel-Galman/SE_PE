\newpage
\section{Proposed Software Architecture}

% ==========================================================
\subsection{System Overview (Gesamtkontext)}

\subsubsection{Übersichtsdiagramm}
\begin{figure}[H]
  \centering
  \includegraphics[width=\textwidth]{diagramms/Overview.png}
  \caption{Gesamtüberblick: Systemkontext und Kommunikations-/Datenflüsse}
  \label{fig:overview_total}
\end{figure}

\subsubsection{Kurzbeschreibung}
Abbildung~\ref{fig:overview_total} ordnet die Emergency-Stop-Funktion in den Gesamtkontext ein.
Sensorik und Hardware-I/O werden auf den Truck-Knoten verarbeitet; die Bedien- und Leitsoftware (Mission Control) kommuniziert
über WLAN mit dem System. Für den Emergency Stop sind insbesondere die Button-/Schalter-Eingänge sowie die Ausgabe an Aktorik
(z.\,B. Bremsanforderung) und Signalisierung relevant.

Zusätzlich sind die internen Knoten wie folgt gekoppelt:
\begin{itemize}[leftmargin=*, itemsep=2pt]
  \item \textbf{Arduino Mega ``Master'' $\leftrightarrow$ Arduino Mega ``Slave'':} Verbindung über \textbf{CAN}.
  \item \textbf{Jetson AGX / Arduino Mega ``Master'' / WLAN-Modul:} Anbindung über internes \textbf{LAN} zum \emph{WLAN Client}.
\end{itemize}

Die Hardware-Anbindung ist wie folgt verteilt:
\begin{itemize}[leftmargin=*, itemsep=2pt]
  \item \textbf{Arduino Mega ``Master'':} \emph{Soft-Stop Schalter} sowie \emph{Lichter/Signalisierung} (z.\,B. Blinklicht).
  \item \textbf{Jetson AGX:} Anbindung der \emph{zwei LiDAR-Sensoren}.
  \item \textbf{Arduino Mega ``Slave'':} \emph{Pilz-Notknopf (Notaus)}.
\end{itemize}

\subsubsection{Subsysteme im Kontext der Emergency-Stop-Funktion}
\begin{figure}[H]
  \centering
  \includegraphics[width=\textwidth]{diagramms/SubSystems.png}
  \caption{Aufteilung und Verbindung der Subsysteme}
  \label{fig:subsystems_context}
\end{figure}

Abbildung~\ref{fig:subsystems_context} zeigt die für den Emergency Stop relevanten Subkomponenten und Schnittstellen:
\emph{TTC-Observer} liefert Informationen/Ereignisse an \emph{Emergency Stop}, das \emph{EnvironmentControl Interface} bindet I/O an,
\emph{DriveControlUnit} setzt Stop-/Fahranforderungen um und \emph{Logging} protokolliert Ereignisse.

% ==========================================================
\subsection{Subsystem Decomposition und Klassendesign}

\subsubsection{Klassendiagramm (Übersicht)}
% Hinweis: Dieses Diagramm deckt die geforderte Klassensicht (Attribute + public Operations) ab.
\begin{figure}[H]
  \centering
  \includegraphics[width=\textwidth]{diagramms/Klassendiagramm_Übersicht.png}
  \caption{Klassendiagramm: Überblick über zentrale Klassen und Zustände}
  \label{fig:class_overview}
\end{figure}

% Optional: wenn ihr die einzelnen Klassendiagramme später separat habt, hier weitere Platzhalter:
% \paragraph{Weitere Klassendiagramme (optional)}
% TODO: ggf. zusätzliche Klassendiagramme je Subsystem einfügen

% ==========================================================
\subsection{Hardware/Software Mapping (Deployment)}
\label{sec:deployment}

\subsubsection{Deployment Diagram}
\begin{figure}[H]
  \centering
  \includegraphics[width=\textwidth]{diagramms/Deployment.png}
  \caption{Zuordnung von Subkomponenten zur Hardware}
  \label{fig:deployment}
\end{figure}

\subsubsection{Zuordnung der Komponenten}
\begin{itemize}[leftmargin=*, itemsep=2pt]
  \item \textbf{Jetson AGX:} \emph{TTC-Observer}
  \item \textbf{Arduino Mega ``Master'':} \emph{EnvironmentControl Interface}
  \item \textbf{Arduino Mega ``Slave'':} \emph{Emergency Stop}, \emph{DriveControlUnit}, \emph{Logging}
\end{itemize}

\subsubsection{Begründung der Verteilung}
\begin{itemize}[leftmargin=*, itemsep=2pt]
  \item \textbf{Arduino Mega ``Slave'':} sicherheitskritische Funktionen + Logging lokal.
  \item \textbf{Arduino Mega ``Master'':} kapselt Hardware-I/O (Interface-Schicht).
  \item \textbf{Jetson AGX:} TTC-Auswertung (\emph{TTC-Observer}) und Weitergabe an Emergency Stop.
\end{itemize}

% ==========================================================
\subsection{Global Software Control}

\subsubsection{Sequence Diagram: TTC}
\begin{figure}[H]
  \centering
  \includegraphics[width=\textwidth]{diagramms/Seq_TTC.png}
  \caption{Sequence Diagram: Interaktion TTC-Observer $\rightarrow$ TTC Handler}
  \label{fig:seq_ttc}
\end{figure}

\subsubsection{Sequence Diagram: EnvironmentControl Interface}
\begin{figure}[H]
  \centering
  \includegraphics[width=\textwidth]{diagramms/Seq_EnvironmentControl_Interface.png}
  \caption{Sequence Diagram: Abfragen von Button-States und Setzen von LEDs}
  \label{fig:seq_envctrl}
\end{figure}

\subsubsection{Sequence Diagram: Drive Control Unit}
\begin{figure}[H]
  \centering
  \includegraphics[width=\textwidth]{diagramms/Seq_Drive_Controll_Unit.png}
  \caption{Sequence Diagram: Stop/Enable/Speed via Drive Controller Handler}
  \label{fig:seq_drive}
\end{figure}

\subsubsection{Sequence Diagram: Logging}
\begin{figure}[H]
  \centering
  \includegraphics[width=\textwidth]{diagramms/Seq_Sequenz_Logging.png}
  \caption{Sequence Diagram: Logging von StateChange, Error und Trace}
  \label{fig:seq_logging}
\end{figure}

% ==========================================================
\subsection{Boundary Conditions}

\subsubsection{Sequence Diagram: Start-up}
\begin{figure}[H]
  \centering
  \includegraphics[width=\textwidth]{diagramms/Seq_Startup.png}
  \caption{Sequence Diagram: Start-up / Initialisierung}
  \label{fig:seq_startup}
\end{figure}

\subsubsection{Sequence Diagram: Shutdown}
\begin{figure}[H]
  \centering
  \includegraphics[width=\textwidth]{diagramms/Seq_ShutDown.png}
  \caption{Sequence Diagram: Shutdown}
  \label{fig:seq_shutdown}
\end{figure}

\subsubsection{Sequence Diagram: Error Scenario}
\begin{figure}[H]
  \centering
  \includegraphics[width=\textwidth]{diagramms/Seq_Error1.png}
  \caption{Sequence Diagram: Error Scenario (Beispiel)}
  \label{fig:seq_error}
\end{figure}