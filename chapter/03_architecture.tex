% ==========================================================
% KAPITEL 3: Proposed Software Architecture (konsolidiert)
% - enthält Context Diagram + Subsystem Decomposition logisch hintereinander
% - Class Diagrams stehen NACH der Subsystem-Aufteilung (passt besser)
% - Pfade/Dateinamen wie bei dir: diagramms/...
% ==========================================================

\section{Proposed Software Architecture}

% ----------------------------------------------------------
\subsection{Overview / Context}

\subsubsection{Context Diagram}
% HINWEIS: Dieses Kapitel beschreibt die Systemgrenze und alle externen Nachbarn.
% Das zentrale Systemelement ist die Komponente "Emergency Stop".
% Extern angebunden sind Mission Control, TTC, Logging, Drive Control Unit sowie Hardware (Buttons/LEDs).

\begin{figure}[H]
  \centering
  \includegraphics[width=\textwidth]{diagramms/Emergency_Stop_System_Overview.png}
  \caption{Context Diagram / System Overview der Komponente \emph{Emergency Stop}}
  \label{fig:context_emergency_stop}
\end{figure}

\subsubsection{Systemgrenze und externe Schnittstellen}
Die Systemgrenze umfasst ausschließlich die truckseitige Softwarekomponente \emph{Emergency Stop}.
Alle in Abbildung~\ref{fig:context_emergency_stop} dargestellten Nachbarsysteme und Hardware-Komponenten gelten als extern
und werden in diesem Dokument nicht intern entworfen.

\paragraph{Externe Software-/Systemschnittstellen}
\begin{itemize}[leftmargin=*, itemsep=2pt]
  \item \textbf{Mission Control:}
  \emph{Emergency Stop} empfängt Stop-Anforderungen über \emph{Stop Request In} via \emph{IE\_StopRequest}
  und liefert Status-/Stop-Informationen über \emph{Stop Information Out} via \emph{IE\_StopInformation}.
  \item \textbf{TTC:}
  Die \emph{Time-To-Collision} wird als Entscheidungsgrundlage über \emph{Distance In} via \emph{IE\_TTC\_Information} bereitgestellt.
  \item \textbf{Logging:}
  Stop-relevante Ereignisse werden über \emph{Logging Out} via \emph{IE\_Logging\_Information} an das externe Logging übergeben.
  \item \textbf{Drive Control Unit:}
  Brems-/Stop-Anforderungen werden über \emph{Braking Out} via \emph{IE\_Stop\_Requestion} an die \emph{Drive Control Unit} ausgegeben.
\end{itemize}

\paragraph{Externe Hardware}
\begin{itemize}[leftmargin=*, itemsep=2pt]
  \item \textbf{Buttons:}
  Hardware-Buttonereignisse (u.\,a. Pilzbutton und Soft-Stop Button) werden über \emph{Button In} via \emph{IE\_ButtonEvent} an
  \emph{Emergency Stop} übergeben.
  \item \textbf{LEDs:}
  Zustände für Warnleuchten und Soft-Stop LED werden über \emph{LED Out} via \emph{IE\_LED\_Status} an die Hardware ausgegeben.
\end{itemize}

\subsubsection{Kurze Kommunikationslogik im Kontext}
Die Komponente \emph{Emergency Stop} nimmt Stop-relevante Eingaben (Button-Events, Stop-Requests und TTC-Informationen) entgegen,
führt die zustandsbasierte Entscheidungslogik aus und gibt daraus resultierende Aktionen nach außen weiter
(Brems-/Stop-Anforderung an die \emph{Drive Control Unit}, Status an \emph{Mission Control}, LED-Zustände an die Hardware sowie Logging).

% ----------------------------------------------------------
\subsection{Subsystem Decomposition}

\subsubsection{Subsystem Decomposition Diagram (Package/Component)}
% HINWEIS: Dieses Diagramm zeigt die interne Aufteilung der Komponente "Emergency Stop" in Subsysteme und deren Interfaces.

\begin{figure}[H]
  \centering
  \includegraphics[width=\textwidth]{diagramms/Emergecy_Stop_Subsystems.png}
  \caption{Subsystem Decomposition der Komponente \emph{Emergency Stop}}
  \label{fig:subsystems_emergency_stop}
\end{figure}

\subsubsection{Aufbau und Verantwortlichkeiten}
Abbildung~\ref{fig:subsystems_emergency_stop} zeigt die interne Aufteilung der Komponente \emph{Emergency Stop} in Subsysteme.
Zentrales Orchestrierungselement ist die \emph{State Machine}, die externe Ereignisse aufnimmt, Zustandsübergänge ausführt und
Folgeaktionen (Stop-Anforderung, Signalisierung, Logging, Fehlerbehandlung und externe Kommunikation) koordiniert.

\paragraph{State Machine (zentrales Steuerungselement)}
\begin{itemize}[leftmargin=*, itemsep=2pt]
  \item Nimmt Events und Requests entgegen (z.\,B. \emph{IE\_Button\_Press}, \emph{IE\_Stop\_request}, \emph{IE\_State\_Information}).
  \item Führt die Zustandslogik für \emph{Hard Stop} und \emph{Soft Stop} aus (inkl. Übergänge und Lösen).
  \item Orchestriert Ausgaben (Stop-Anforderung, LED-Indikation) sowie Logging und Fehlerbehandlung.
\end{itemize}

\paragraph{Communication Handler}
\begin{itemize}[leftmargin=*, itemsep=2pt]
  \item Kapselt die Kommunikation mit \emph{Mission Control}.
  \item Übergibt eingehende Stop-Requests an die \emph{State Machine}.
  \item Übermittelt Zustands-/Stop-Informationen nach außen (Status/Stop-Information).
\end{itemize}

\paragraph{Emergency Stop Logic}
\begin{itemize}[leftmargin=*, itemsep=2pt]
  \item Bewertet \emph{TTC}-Informationen (Input via \emph{IE\_TTC\_Information}) als Entscheidungsgrundlage.
  \item Leitet daraus eine Stop-Entscheidung ab und übergibt diese an die \emph{State Machine} (z.\,B. via \emph{IE\_Stop\_Decision}).
\end{itemize}

\paragraph{Button Handler}
\begin{itemize}[leftmargin=*, itemsep=2pt]
  \item Nimmt Hardware-Buttonereignisse entgegen (Input via \emph{IE\_ButtonEvent}).
  \item Übersetzt die Eingaben in interne Events für die \emph{State Machine} (Output via \emph{IE\_Button\_Press}).
\end{itemize}

\paragraph{LED Handler}
\begin{itemize}[leftmargin=*, itemsep=2pt]
  \item Setzt LED-/Warnleuchten-Anforderungen aus der \emph{State Machine} um (Input via \emph{IE\_LEDIndication}).
  \item Gibt den LED-Status an die Hardware aus (Output via \emph{IE\_LED\_Status}).
\end{itemize}

\paragraph{Logging Handler}
\begin{itemize}[leftmargin=*, itemsep=2pt]
  \item Erzeugt und bündelt stop-relevante Log-Einträge (Auslösen, Zustandswechsel, Lösen).
  \item Übergibt Logging-Informationen an die externe Logging-Schnittstelle (Output via \emph{IE\_Logging\_Information}).
\end{itemize}

\paragraph{Error Handler}
\begin{itemize}[leftmargin=*, itemsep=2pt]
  \item Verarbeitet Fehler-/Ausnahmesituationen (z.\,B. aus Zustandswechseln oder Schnittstellenfehlern).
  \item Unterstützt definierte Reaktionen in Fehlerfällen (z.\,B. Logging, Statusmeldung, definierte Fallbacks gemäß SRS).
\end{itemize}

\subsubsection{Interaktionsprinzip}
Die Interaktion folgt dem Muster: Eingaben (Buttons/Requests/TTC) werden über spezialisierte Subsysteme aufgenommen und an die
\emph{State Machine} überführt. Die \emph{State Machine} trifft die zustandsbasierte Entscheidung und stößt anschließend die
erforderlichen Aktionen an (Stop-Anforderung, LED-Indikation, Logging und Fehlerbehandlung sowie externe Kommunikation).

% ----------------------------------------------------------
\subsection{Class Diagrams je Subsystem}
% TODO: HIER für JEDES Subsystem ein Class Diagram einfügen (Attribute + public operations).
% HINWEIS: Die folgenden Platzhalter bitte durch eure exportierten Diagramme ersetzen.

\paragraph{EmergencyStopper / Emergency Stop Logic -- Class Diagram}
% TODO: Diagramm einfügen
% \begin{figure}[H]
%   \centering
%   \includegraphics[width=\textwidth]{diagramms/class_emergency_stop_logic.png}
%   \caption{Class Diagram: Emergency Stop Logic}
% \end{figure}

\paragraph{State Machine -- Class Diagram}
% TODO: Diagramm einfügen

\paragraph{Communication Handler -- Class Diagram}
% TODO: Diagramm einfügen

\paragraph{Button Handler -- Class Diagram}
% TODO: Diagramm einfügen

\paragraph{LED Handler -- Class Diagram}
% TODO: Diagramm einfügen

\paragraph{Logging Handler -- Class Diagram}
% TODO: Diagramm einfügen

\paragraph{Error Handler -- Class Diagram}
% TODO: Diagramm einfügen

% ----------------------------------------------------------
\subsection{Architekturentscheidungen und Alternativen}
% TODO:
% - Warum diese Subsystem-Aufteilung?
% - Welche Alternativen wurden erwogen (z.B. alles in einer Klasse vs. Handler-Aufteilung)?
% - Technologieentscheidung (C vs. C++) und Konsequenzen für OO-Modellierung.

% ----------------------------------------------------------
\subsection{Hardware/Software Mapping}

\subsubsection{Deployment Diagram}
% TODO: HIER Deployment Diagram einfügen.
% Mapping: welche Software-Subsysteme laufen auf welcher Hardware/ECU/Node
% + externe Hardware (Buttons/LEDs) als Peripherie.
% \begin{figure}[H]
%   \centering
%   \includegraphics[width=\textwidth]{diagramms/deployment_diagram.png}
%   \caption{Deployment Diagram (Emergency Stop)}
% \end{figure}

% ----------------------------------------------------------
\subsection{Global Software Control}

\subsubsection{Sequence Diagram: Emergency Stop (Interaktion)}
% TODO: HIER mindestens 1 Sequence Diagram einfügen, das den Emergency-Stop-Ablauf zeigt.
% Muss Klassen/Interfaces aus euren Class Diagrams wiederverwenden
% (z.B. Trigger Pilzbutton -> Hard Stop; Pilz lösen -> Soft Stop; TTC -> Soft Stop; Logging immer).
% \begin{figure}[H]
%   \centering
%   \includegraphics[width=\textwidth]{diagramms/seq_emergency_stop.png}
%   \caption{Sequence Diagram: Emergency Stop}
% \end{figure}

% ----------------------------------------------------------
\subsection{Boundary Conditions}

\subsubsection{Sequence Diagram: Start-up}
% TODO: HIER Sequence Diagram für Start-up einfügen.

\subsubsection{Sequence Diagram: Shutdown}
% TODO: HIER Sequence Diagram für Shutdown einfügen.

\subsubsection{Sequence Diagram: Error Scenario}
% TODO: HIER Sequence Diagram für ein Fehler-/Ausnahmeszenario einfügen.